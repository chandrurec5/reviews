%%%%%%%%%%%%%%%%%%%%%%%%%%%%%%%%%%%%%%%%%%%%%%%%%%%%%%%%%%%%%%%%%%
%%%%%%%% ICML 2016 EXAMPLE LATEX SUBMISSION FILE %%%%%%%%%%%%%%%%%
%%%%%%%%%%%%%%%%%%%%%%%%%%%%%%%%%%%%%%%%%%%%%%%%%%%%%%%%%%%%%%%%%%

% Use the following line _only_ if you're still using LaTeX 2.09.
%\documentstyle[icml2016,epsf,natbib]{article}
% If you rely on Latex2e packages, like most moden people use this:
\documentclass{article}

% use Times
\usepackage{times}
% For figures
\usepackage{graphicx} % more modern
%\usepackage{epsfig} % less modern
\usepackage{subfigure} 

% For citations
\usepackage{natbib}
\usepackage{amsmath}
\newcommand{\R}{\mathbf{R}}
% For algorithms
\usepackage{algorithm}
\usepackage{algorithmic}

% As of 2011, we use the hyperref package to produce hyperlinks in the
% resulting PDF.  If this breaks your system, please commend out the
% following usepackage line and replace \usepackage{icml2016} with
% \usepackage[nohyperref]{icml2016} above.
\usepackage{hyperref}

% Packages hyperref and algorithmic misbehave sometimes.  We can fix
% this with the following command.
\newcommand{\theHalgorithm}{\arabic{algorithm}}

% Employ the following version of the ``usepackage'' statement for
% submitting the draft version of the paper for review.  This will set
% the note in the first column to ``Under review.  Do not distribute.''
\usepackage{icml2016} 

% Employ this version of the ``usepackage'' statement after the paper has
% been accepted, when creating the final version.  This will set the
% note in the first column to ``Proceedings of the...''
%\usepackage[accepted]{icml2016}


% The \icmltitle you define below is probably too long as a header.
% Therefore, a short form for the running title is supplied here:
\icmltitlerunning{Paper 1822}

\begin{document} 

\onecolumn
\icmltitle{Paper $1822$: Concentration Bounds for Two Timescale Stochastic Approximation with Applications to Reinforcement Learning}

% It is OKAY to include author information, even for blind
% submissions: the style file will automatically remove it for you
% unless you've provided the [accepted] option to the icml2016
% package.
\icmlauthor{Your Name}{email@yourdomain.edu}
\icmladdress{Your Fantastic Institute,
            314159 Pi St., Palo Alto, CA 94306 USA}
\icmlauthor{Your CoAuthor's Name}{email@coauthordomain.edu}
\icmladdress{Their Fantastic Institute,
            27182 Exp St., Toronto, ON M6H 2T1 CANADA}

% You may provide any keywords that you 
% find helpful for describing your paper; these are used to populate 
% the "keywords" metadata in the PDF but will not be shown in the document
\icmlkeywords{boring formatting information, machine learning, ICML}
\vskip 0.3in


Consider the stochastic approximation given by
\begin{align}\label{sa}
\begin{split}
\theta_{n+1}=\theta_n+\alpha_n(0 -q\theta_n),\\
w_{n+1}=w_n+\beta_n(0 -w_n),
\end{split}
\end{align}
with step sizes $\alpha_n=\frac{1}{n}$, say $0<q<0.1$ and $\theta_n\in \R$, $w_n\in \R$, $\beta_n=\frac{1}{\sqrt{n}}$. We can put \eqref{sa} into the framework in the paper by setting $h_1=0-0.1\theta$, $h_2=0-1\,w$, $M^{(i)}_{n+1}=0,\,i=1,2$. In this case $\theta^*=0$ and $\lambda(\theta)=0$. The related ordinary differential equation is then $\dot{\theta}(t)=-q\theta(t)$ whose solution is $\theta(t)=\theta(0)e^{-qt}$ with probability $1$.  Assume $\theta(0)=\theta_0=1$ for the rest of the discussion. For the aforementioned step size $t(n)=\sum_{0\leq k\leq n} \frac{1}{k}\approx \log n$ and hence
\begin{align}\label{loose}
\theta(t) =e^{-q t} \Rightarrow \theta_n \approx e^{-q \log n} = \frac{1}{n^{q}}
\end{align}
Thus the eigenvalue shows up in the rate in the exponent in \eqref{loose}. A tighter analysis can be done on the original sequence say for $n>n_0$
\begin{align*}
\theta_{n}&=\Pi_{k=0}^{n-n_0}(1-\frac{q}{k+n_0}) \theta_{n_0}\\
|\theta_{n}|&=\Pi_{k=0}^{n-n_0}(1-\frac{q}{k+n_0}) |\theta_{n_0}|
\end{align*}
Now using the fact that for $n_0$ sufficiently large $e^{-2\frac{q}{k+n_0}}\leq(1-\frac{q}{k+n_0})\leq e^{-\frac{q}{k+n_0}}$ we have
\begin{align}\label{exact}
 |\theta_{n+n_0}|&\geq |\theta_{n_0}|e^{-2q\sum_{k=0}^{n-n_0}\frac{1}{k+n_0} }\\
|\theta_{n+n_0}|&\geq |\theta_{n_0}|e^{-2q\log(n-n_0) } \\
|\theta_{n+n_0}|&\geq |\theta_{n_0}|\big(\frac{n_0}{n}\big)^{2q}
\end{align}
and the eigenvalue shows up in the form as $\frac{1}{n^{2q}}$.

We now apply Corollary~$2$ to the stochastic approximation in \eqref{sa} by setting $\epsilon_2=\infty$. Now we have
\begin{align*}
Pr\{ |\theta_n|>\epsilon_1|G'_{n_0}\}\leq C[e^{-n_0c'_1\epsilon_1^2}+e^{-\sqrt{n_0}c'_2\epsilon_1^2/\log(n_0)}]
\end{align*}
Considering the fact that out of the two exponents on the right hand side of the above equation, the second term is dominant for large $n_0$ we have
\begin{align*}
Pr\{ \left|\theta_n\right|>\epsilon_1|G'_{n_0}\}\leq C[e^{-n_0c'_1\epsilon_1^2}+e^{-\sqrt{n_0}c'_2\epsilon_1^2/\log(n_0)}]\approx Ce^{-\sqrt{n_0}c'_2\epsilon_1^2/\log(n_0)}
\end{align*}
Now let $\delta=Ce^{-\sqrt{n_0}c'_2\epsilon_1^2/\log(n_0)}$,
\begin{align}\label{interval}
Pr\Bigg\{ \left|\theta_n\right|> \sqrt{\frac{{\log(n_0) \log(\frac{C}{\delta})}}{{c'_2} \sqrt{n_0}} }\Bigg|G'_{n_0}\Bigg\}\leq \delta.
\end{align}
Since $n\geq n_1\geq n_0$, we have equation \eqref{interval} starts holding for all $n\geq n_1\geq n_0$ with high probability (i.e., $1-\delta$). So for sufficiently large $n\geq n_0$, $\theta_n$ reaches $0$ at rate roughly $\frac1{n_0^{\frac{1}{4}}}$. However, from \eqref{loose} and \eqref{exact} we expect a rate $\frac1{n^q}$ where $q$ is the eigenvalue, which can be arbitrarily small.

\textbf{Issue:} $c'_1$ and $c'_2$ (in Corollary~$2$ of the paper) might be dependent on $q_1$ and $q_2$ which are actually the minimum eigenvalues as mentioned in lines $489-490$ above equation (28) of the submission. However, $c'_1$ and $c'_2$ are not showing up in the exponent of $n_0$ in \eqref{interval}.

\end{document}






